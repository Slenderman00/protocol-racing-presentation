\documentclass[aspectratio=169]{beamer}
\usetheme{Madrid}

\usepackage[utf8]{inputenc}
\usepackage[T1]{fontenc}
\usepackage{lmodern}
\usepackage{graphicx}
\usepackage{booktabs}
\usepackage{hyperref}

\title[Protocol Racing]{Protocol Racing}
\subtitle{Something Something Something}
\author[Joar Heimonen]{}
\date{\today}

\begin{document}

\begin{frame}
  \titlepage
\end{frame}

\begin{frame}{Agenda}
  \tableofcontents
\end{frame}

\section{A bit of history}
\begin{frame}{IPv4}
  \begin{itemize}
    \item \textbf{RFC 791} – \emph{Internet Protocol}
    \item Written for DARPA in 1981 (before the IETF existed)
    \item Designed to interconnect different packet-switched networks (ARPANET, SATNET, university nets)
    \item Created under the assumption that every device would have its own globally unique, routable address
    \item 32-bit address space — \(2^{32} = 4{,}294{,}967{,}296\) possible addresses
    \item Sounds like a lot\ldots\ until you remember that “\textbf{3 billion devices run Java}”
  \end{itemize}
\end{frame}

\section{Main}
\begin{frame}{Key Idea}
  \begin{block}{Takeaway}
    Keep each slide focused on one idea.
  \end{block}
\end{frame}

\section{Wrap-up}
\begin{frame}[standout]
  Questions?
\end{frame}

\end{document}
